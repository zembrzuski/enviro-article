\documentclass[10pt, conference]{IEEEtran}

\usepackage[utf8]{inputenc}
\inputencoding{utf8}
% *** GRAPHICS RELATED PACKAGES ***
\usepackage{subimages}
\setfigdir{figs}

% *** MATH PACKAGES ***
\usepackage[cmex10]{amsmath}
\interdisplaylinepenalty=2500
\usepackage{amsthm}
\newtheorem{definition}{Definition}

% *** PDF, URL AND HYPERLINK PACKAGES ***
\usepackage{hyperref}


\usepackage{listings}
\usepackage{color}

\definecolor{dkgreen}{rgb}{0,0.6,0}
\definecolor{gray}{rgb}{0.5,0.5,0.5}
\definecolor{mauve}{rgb}{0.58,0,0.82}

\lstset{frame=tb,
  language=Perl,
  aboveskip=3mm,
  belowskip=3mm,
  showstringspaces=false,
  columns=flexible,
  basicstyle={\small\ttfamily},
  numbers=none,
  numberstyle=\tiny\color{gray},
  keywordstyle=\color{blue},
  commentstyle=\color{dkgreen},
  stringstyle=\color{mauve},
  breaklines=true,
  breakatwhitespace=true,
  tabsize=3
}


\hyphenation{op-tical net-works semi-conduc-tor}


\begin{document}

%
% paper title
% can use linebreaks \\ within to get better formatting as desired
\title{Envirocar visualization \\ exploring an environmental and traffic data set}

%------------------------------------------------------------------------- 
% change the % on next lines to produce the final camera-ready version 
\newif\iffinal
%\finalfalse
\finaltrue
\newcommand{\jemsid}{99999}
%------------------------------------------------------------------------- 

% author names and affiliations
% use a multiple column layout for up to two different
% affiliations

\iffinal
  \author{%
    \IEEEauthorblockN{Rodrigo Claro Zembrzuski}
    \IEEEauthorblockA{%
      Programa de Pós Graduação em Ciência da Computação\\
      Universidade Federal do Rio Grande do Sul - UFRGS\\
      Porto Alegre, Brasil\\
      Web page: http://inf.ufrgs.br/$\sim$rczembrzuski}
  }
\else
  \author{Sibgrapi paper ID: \jemsid \\ }
\fi


% make the title area
\maketitle


\begin{abstract}
Envirocar is a project that collects and stores traffic and environment data for
sustainable mobility. 

They are feeded by sensors deployed within vehicles, that brings accurate information
about the car navigation and environmental data, such as the emission of carbon dioxide 
and fuel consumption of the vehicle.

This article aims to characterize the dataset and presents a series of
views for exploratory analysis of this database. With this, it will be
obtained intuition about the type of data presented. In addition,
accurate information on the feasibility of using this database for modeling
machine learning algorithms.



% DO NOT USE SPECIAL CHARACTERS, SYMBOLS, OR MATH IN YOUR TITLE OR ABSTRACT.
%
\end{abstract}

\begin{IEEEkeywords}
envirocar; smartcars; kibana; geolocation; data visualization;

\end{IEEEkeywords}


\IEEEpeerreviewmaketitle


% Wherever Times is specified, Times Roman or Times New Roman may be used. If neither is available on your system, please use the font closest in appearance to Times. Avoid using bit-mapped fonts if possible. True-Type 1 or Open Type fonts are preferred. Please embed symbol fonts, as well, for math, etc.

%==========================================
%==========================================


%==========================================
\section{Introduction}
Envirocar é um projeto da Universidade de Münster. O objetivo 
do projeto é oferecer uma plataforma simples para coleta, 
armazenamento e distribuição de dados ambientais e dados de logística
de tráfego. 

A plataforma une dados gerados por sensores comuns,
que vem na maioria dos automóveis atuais com sensores de 
geolocalizacao, presente na maioria dos smartphones. Com isso,
cria-se uma base de dados supostamente relevante para que seja medido o fluxo de automóveis
nas avenidas e quantidade de emissão de poluentes nas cidades. 

%------------------------------------------------------------------------- 
\subsection{Motivacao}
%

Um dos grandes desafios da sociedade nos nossos dias é preservar
e desenvolver a mobilidade urbana de maneira sustentável. Isto é,
mitigar o impacto do transporte de passageiros meio-ambiente e no
bem-estar geral das pessoas.

Uma base de dados consistente é crucial para que se confirmem
-- ou refutem -- hipóteses relativas a comportamento no trânsito
em diferentes partes do globo. Além disso, a base consistente
pode ser muito importante para a inferência de padrões comportamento 
e de pilotagem dos motoristas de automóvel. 

Uma análise exploratória é um passo necessário para que se verifique
se a base realmente pode ser utilizada com os propositos citados 
ou se ela deve ser descartada.


%------------------------------------------------------------------------- 
\subsection{Objetivos}
%
O objetivo desse trabalho é fazer uma exploração básica pela base dados
para ganhar intuição sobre ela. Serão observados padrões básicos de 
comportamento em diferentes partes do globo e tambem padrões individuais
de pilotagem para determinados motoristas.

Com isso, será determinado se o uso dessa base pode ser aconselhável
em experimentos mais complexos, como 
treinamento de modelos de aprendizagem. 

%------------------------------------------------------------------------- 
\paragraph*{Contribuições}
%
O trabalho dará uma sugestão sobre a possibilidade de uso da base de dados
envirocar em projetos de aprendizagem de máquina. Tambem será feita uma
análise comparativa entre diferentes técnicas de visualização dos dados
coletados.


%==========================================
\section{Caracterização dos dados}
%

Os dados são obtidos através de sensores comuns em diversos
automóveis (OBD-II) e enriquecidos com informações de geolocalizacao
que estao presentes em smartphones. Para preservar a identidade
dos usuarios, suas informações pessoais nao são disponibilizadas e tambem
são retiradas as informações dos primeiros e ultimos 200m de deslocamento,
com objetivo que o usuario nao seja identificado através das informações
de origem e de destino. 

Uma vez que os automóveis estao em deslocamento, suas informações mudam
rapidamente. Desse modo, informações populadas no dataset a cada 2 segundos.

Os dados originais do servidor são persistidos num banco de dados NoSQL -- 
MongoDB -- e oferecidos abertos ao publico através de um RESTful web service.
Os dados são oferecidos em um formato JSON, que é simples de entender e processar.

Para realizacao desse trabalho, foram acessadas as apis rest e persistidas
as informações de cada deslocamento no banco de dados ElasticSearch, por 
motivos que serão esclarecidos em seguida. 



\subsection{Caracterização geral dos dados}

Cada item do dataset é composto por dois atributos principais, que são subdivididos
em diversas partes e serão mais detalhados na sequencia: 'properties' e 'features'. 

\begin{itemize}

  \item properties: possui as caracteristicas gerais do veículo em questao. Esta subdividido
  nos itens apresentados na tabela 1. Veja um exemplo de propriedades extraido do banco de dados.


\begin{lstlisting}
"properties": {
  "sensor": {
    "type": "car",
    "properties": {
      "constructionYear": 2011,
      "model": "Avensis",
      "fuelType": "gasoline",
      "id": "574e78cbe4b09078f97bbb4a",
      "engineDisplacement": 1800,
      "manufacturer": "Toyota"
    }
}
\end{lstlisting}



\begin{table}
\caption{Properties: propriedades gerais de cada veículo}
\label{tab:perfs}
\centering
\begin{tabular}{l|c}
\multicolumn{1}{l|}{\bf Variavel} &
\multicolumn{1}{l}{\bf Descricao}  \\ \hline
type                & categorica       \\
constructionYear    & discreta           \\
model               & categorica         \\
fuelType            & categorica           \\
engineDisplacement  & discreta             \\
manufactuerer       & categorica       
\end{tabular}
\end{table}

  \item features: possui as caracteristicas do deslocamento -- ou da 'viagem' em si. A viagem
  é caracterizada pela composicao, a cada dois segundos, do seguinte conjunto de dados:



\begin{table}
\caption{Features: informações de cada timestamp do deslocamento}
\label{tab:perfs}
\centering
\begin{tabular}{l|c}
\multicolumn{1}{l|}{\bf Variavel} &
\multicolumn{1}{l}{\bf Unidade de medida}  \\ \hline
coordinates        & geoespacial       \\
speed              & continuo         \\
rpm                & continuo       \\
gps accuracy       & continuo       \\
maf                & continuo         \\
engine load        & continuo       \\
gps pdop           & continuo       \\
o2 lambda voltage  & continuo       \\
throttle position  & continuo       \\
consumption        & continuo       \\
gps vdop           & continuo       \\
gps speed          & continuo       \\
gps bearing        & continuo       \\
intake pressure    & continuo       \\
co2                & continuo       \\
time               & temporal       \\

\end{tabular}
\end{table}








\begin{lstlisting}
{
        "geometry": {
          "coordinates": [
            6.443663779195363,
            51.20348336793408
          ],
          "type": "Point"
        },
        "type": "Feature",
        "properties": {
          "phenomenons": {
            "Speed": {
              "value": 34.647194623947144,
              "unit": "km/h"
            },
            "Rpm": {
              "value": 1584.3050694465637,
              "unit": "u/min"
            },
            "GPS Accuracy": {
              "value": 2.999999910593033,
              "unit": "%"
            },
            "MAF": {
              "value": 9.437765815629945,
              "unit": "l/s"
            },
            "Engine Load": {
              "value": 43.96216858346017,
              "unit": "%"
            },
            "GPS PDOP": {
              "value": 1.4999999776482582,
              "unit": "precision"
            },
            "O2 Lambda Voltage": {
              "value": 3.2762881521175586,
              "unit": "V"
            },
            "Throttle Position": {
              "value": 21,
              "unit": "%"
            },
            "Consumption": {
              "value": 3.102402130874109,
              "unit": "l/h"
            },
            "GPS VDOP": {
              "value": 1.1779116287827491,
              "unit": "precision"
            },
            "GPS Speed": {
              "value": 33.47147411240462,
              "unit": "km/h"
            },
            "GPS HDOP": {
              "value": 0.899999986588955,
              "unit": "precision"
            },
            "Intake Pressure": {
              "value": 44.14216932654381,
              "unit": "kPa"
            },
            "GPS Bearing": {
              "value": 304.7174273121018,
              "unit": "deg"
            },
            "Intake Temperature": {
              "value": 13.000000387430191,
              "unit": "c"
            },
            "CO2": {
              "value": 7.290645007554156,
              "unit": "kg/h"
            },
            "O2 Lambda Voltage ER": {
              "value": 0.9988191702782387,
              "unit": "ratio"
            },
            "GPS Altitude": {
              "value": 104.90115790988267,
              "unit": "m"
            }
          },
          "id": "590ad752268d1b08a47f18d4",
          "time": "2017-03-27T04:51:05Z"
        }
      }
\end{lstlisting}



\end{itemize}




\subsection{Questoes a serem respondidas}

A ideia básica é responder questoes relativas a (a) padrões de motoristas
em diferentes partes do globo, (b) padrões de comportamento individual no
volante e (c) possibilidade de uso da base em modelos de aprendizagem de máquina.

\begin{itemize}
  \item Existem muitos registros nesse banco de dados?
  \item Em quer periodo de tempo esse banco de dados foi utilizado? Ainda hoje ele é bastante utilizado?
  \item Em que regioes do globo estao os usuarios desse sistema?
  \item Os modelos e fabricantes mais comum na Europa são os mesmos do Brasil?
  \item Quais marcas e modelos consomem mais combustivel?
  \item Quais os modelos e fabricantes mais comuns?
  \item Quais são as marcas e modelos que mais poluem? quais são as marcas e modelos que menos poluem?
  \item Podemos descobrir quais são as regioes mais poluidas e regioes menos poluidas?
  \item é possivel utilizar este dataset para extrair padrões de comportamento de um determinado usuario?
\end{itemize}


%==========================================
\section{Preparacao dos dados para visualicao}

O envirocar é uma base aberta de dados. Entretanto, nao existe uma forma de simples 
para fazer download da base inteira. A entrega dos dados é feita via servicos REST
que possibilitam acesso a determinadas partes do banco de dados.

Para que se realizasse a obtencao da base inteira para que pudessem ser feitas as 
visualizações, utilizou-se um script capaz de varrer a base de maneira sequencial. 
Esse mesmo script é responsavel pela persistencia dos dados no elasticsearch.

A base de dados possui 15 mil viagens. O numero bruto nao é tao grande quanto se supunha,
mas, para cada vaigem, é enviado ao banco de dados uma serie de informações a cada dois
segundos. Dessa maneira, embora o numero de viagens seja relativamente 
pequeno, o tamanho de cada viagem é grande. A indexada no elasticsearch 
ocupou 6 Gb de memoria em disco.



%==========================================
\section{técnica de visualização desenvolvida}
\label{sec:technique}
%

A exploração dos dados foi feita com tres abordagens. Uso de Kibana, linguagem R
e API de mapas do Google.

Kibana é uma ferramenta utilizada para análise por inspecao manual e visualização
de informações que funciona de maneira natural com o ElasticSearch. Dessa  forma, 
é  ela  que ira apresentar os dados armazenados no ElasticSearch, em uma interface,  via
browser, altamente customizável com histogramas, mapas e outros painéis que propiciam uma visão
geral sobre os dados.  O Kibana possibilita transformar os logs em informações úteis (valor)
através  de Dashboards,  pois  permite  realizar  correlação  de  eventos,  filtrar logs
por  origem, hospedeiros, entre outras combinações.

Foi utilizado o ElasticSearch como base primaria dos dados, pois Além do suporte a persistencia,
vem junto com uma serie de mecanismos e algoritmos de recuperacao de informacao. A ferramenta
permite combinar geolocalizacao com outras técnicas de recuperacao de informações baseadas
em texto.

Outras ferramenta para visualização de informações utilizada foi a linguagem R. A linguagem
tambem possui uma variada gama de técnicas para visualização de dados, Além de possuir um
suporte muito grande a questoes de estatistica e de probabilidade. 

O R nao é uma ferramenta de tao alto nivel de abstracao quanto o kibana em que, uma vez
que os dados estao persistidos, se montam visualizações de maneira simplificada. Em
vez disso, é necessário
que se escreva codigo para que as visualizações aparecam. Os graficos surgem da maneira
mais crua do que no Kibana, mas possui muito mais flexibilidade, maleabilidade
e estensibilidade. Além disso, existe um conjunto muito mais vasto de visualizações
que o kibana, com vieses mais estatisticos, como boxplots e matrizes de correlacoes.

Por fim, para detectar padrões de comportamente individuais, foi implementada utilizando
a API do Google Maps uma ferramenta simples para visualizar por quais lugares um determinado
motorista frequenta com maior frequencia.

%------------------------------------------------------------------------- 
\subsection{Histograma}

Com o intuito de descobrir se existem muitos registros no banco de dados e, sobretudo,
quais são os periodos em que a ferramente envirocarm são mais utilizados, foi desenvolvido
um histograma em que o eixo x corresponde a variavel temporal e o eixo y corresponde
a quantidade de viagens naquele periodo de tempo. Com isso, podera ser inferido se bastante
gente vem utilizando a ferramenta e, principalmente, se se uso vem crescendo ou decrescendo
com o passar do tempo. é um estudo de tendencias, portanto.

Pode-se observar o grafico abaixo, que nao existe uma tendencia de aumento ou de diminuicao
ao longo do tempo. O que se observa é um 'boom' de uso no periodo em volta do ano de 2016,
mas que nao se mantem no restante do tempo. No periodo de novembro de 2016, por exemplo, 
tivemos 3600 viagens naquele mes. Mas no inicio de 2017, nao passavam de 100 viagens por mes.

Nao se pode observar, portanto, um uso massivo da plataforma. Entretanto, temos uma amostra
que parece razoavel para que se possa tentar responder as outras perguntas do artigo.

\oneimage{ No eixo x, observamos a quantidade de uso da plataforma ao longo dos meses.}{.99}{simple_histogram.png}


%------------------------------------------------------------------------- 
\subsection{Heatmaps}

Uma vez visualizada a distribuição temporal da utilizacao da plataforma, ira ser observada a
distribuição geografica dos usuarios. Como ja se sabe que é uma plataforma desenvolvida
na Alémanha, pode-se supor que muita gente na Europa ira utilizar a ferramenta, em relacao
as outras regioes do globo.

A visualização por calor, utilizando como metrica a quantidade de viagens dispostas no globo
terrestre, dará um senso de quais regioes são mais utilizadas a ferramenta.

Observando o grafico a seguir, observamos aquilo que ja era suposto de antemao: de fato,
na Europa, a plataforma é mais utilizada. Além disso, conseguimos obter alguns dados que nao
eram imaginados. Vemos pontos isolados de uso nos Estados Unidos, Brasil, India e Australia.

Com as informações geoespaciais, ganha-se confianca de que poderao ser respondidas outras questoes,
como o comportamento dos motoristas em diferentes regioes do globo. Além disso, pode-se utilizar
esse grafico para clusterizar de maneira macro o globo terrestre de acordo com essa base de 
dados: America do Norte, America do Sul, Europa, India e Australia.

A visualização de heatmat traz uma ideia muito clara sobre os pontos que mais utilizam a 
ferramenta. Traz-nos a ideia de que em diversos pontos do globo ela esta sendo utilizada, com
enfase na Europa. Traz confianca de que poderemos observar padrões de comportamento em diversos
pontos do globo. Ela traz uma ideia básica, mas nao definitiva, da quantidade de pessoas
que utiliza a ferramenta ao longo do globo.


\oneimage{ Aqui, podemos observar que a maioria dos usuarios, de fato, se encontra,
na Europa. Contudo, existem pessoas utilizando em diversos outros pontos do globo também.}{.99}{heatmap.png}


%------------------------------------------------------------------------- 
\subsection{Grafico em pizza}

O heatmap deu uma ideia básica sobre a quantidade de pessoas que utilizam a plataforma ao redor
do globo. Entretanto, nao trouxe uma ideia definitiva sobre as proporcoes de cada regiao.

Para isso, será utilizado um grafico em pizza, que traz claramente a ideia de proporcoes ao
usuario, ao dispor em fatias cada segmento analizado.

Nesse grafico, será utilizado o cluster inferido visualmente do heatmatp com as seguintes
regioes: Europa, America do Norte, America do Sul, India e Australia.

Aqui, pode-se observar a maioria, de fato, utilizando na Europa: 97\% dos usuarios
esta la. Temos alguma representatividade, com 1\% dos Usuarios na America do Norte e 1\% 
na Australia.

Na America do Sul e na India, nao chega a 1\% de usuarios. Parece ter sido realmente um usuario
curioso que comecou a utilizar a plataforma em sua terra natal.

Essa informacao podera ser utilizada da seguinte maneira: para comparacao de
dados globais, como estatisticas e tentativas de deteccao de comportamento em massa, devera
ser dado foco nos usuarios da Europa.

Entretanto, essa falta de representatividade em outras partes do globo podera favorecer um
olhar mais individualista. Por exemplo, pode-se acreditar que na America do Sul e na India,
é sempre a mesma pessoa que utiliza a ferramenta. Desse modo, poderemos observar algum
comportamento individual. Do mesmo modo pode ser feito na America do Norte e na Australia,
com um pouco mais de criterio, pois la houve muito mais uso.

Entretanto, na Europa, nao podemos fazer isso, pois é muita gente usando. Na Europa, serão
feitas análises mais genericas.

\oneimage{ 98\% dos deslocamentos esta na Europa. 1\% esta na America do Norte e 1\% esta
na Australia. Outras regioes são são insignificantes, que nem visualizamos no grafico .}{.99}{pizza_por_regiao.png}



%------------------------------------------------------------------------- 
\subsection{Graficos em pizza}

Observadas algumas questoes temporais e espaciais, irao ser analisados alguns padrões de comportamento
em diferentes regioes do globo. será utilizado aquele cluster inferido do primeiro heatmap para
fazer uma análise sobre quais são os fabricantes mais utilizados em diferentes partes do globo.

Sobre padrões de comportamento, espera-se verificar se supostos padrões culturais podem de uma determinada
regiao do globo pode ser inferida através deste banco de dados. Por exemplo, é verdade que americanos
gostam de carros espacosos? é verdade que na Europa e Estados Unidos os carros são muito melhores
do que os da America do Sul?

Para responder essa pergunta, algumas visualizações foram feitas algumas suposicoes de técnicas
de visualização: sunbsurst e nugget. 

A técnica de sunbsurst nao pareceu muito adequada devido a quantidade de informações -- muito grande para
um espaco pequeno -- e o biscoito nao transpareceu de maneira clara o que estava sendo perguntado.

Para isso, foi utilizada novamente a técnica de visualização de graficos em pizza. Entretanto, agora,
em vez de plotar somente um grafico, o foram plotados 5 graficos em pizza, em que cada pizza corresponde
a uma regiao do cluster. Essa visualização foi escolhida em detrimento do sunburst pois a segmentacao
trouxe mais evidencia para respoder aquilo que estava sendo perguntado. O sunburst trouxe muita informacao
em muito pouco espaco fisico, o que pareceu meio confuso.

Observando os fabricantes, vemos que as principais marcas na America do Norte, Australia e Europa
são as mesmas.

\begin{itemize}
  \item America do Norte: Chevrolet, Toyota e Hiunday
  \item Australia: Ford, Toyota e Opel
  \item Europa: Toyota, Ford, Opel, VW, BMW
\end{itemize}

Ja na America do Sul e India, parecem ser regioes mais humildes tambem, pois aqui nao observamos
Hiunday e BMW.

\begin{itemize}
  \item America do Sul: Ford
  \item India: Renault e VW
\end{itemize}

Baseado nessa amostra de dados, nao se pode chegar a uma conclusão definiva para as
questoes levantadas. A distribuição entre Europa, Estados Unidos e Australia ficaram razoavelmente
parecidas. Ja na America do Sul e India, realmente parecem regioes menos desenvolvidas. Entretanto,
a amostra dos dados nao parece ser significativa o suficiente para uma conclusão assertiva. A observacao
do grafico parece corrobarar a ideia inicial, utilizada para fazer as pergutnas, mas nao ha evidencias
que, de fato, elas acontecem.

\oneimage{ Estados Unidos e Europa apresentam resultados parecidos. America do Sul supoe que somente um cara utilizou a plataforma.}{.99}{marcas_por_regiao.png}


%------------------------------------------------------------------------- 
\subsection{Sunburst}

A visualização por sunburst é uma técnica de visualização radial apra exibir estruturas hierarquicas
como arvores. Essa técnica mostra-se muito adequada para condensar informações hierarquicas, 
como para responder a pergunta:
para cada fabricante, qual é o modelo de automóvel mais utilizado. 

Como na Europa existe uma amostra muito mais significicativa do que em outras regioes do globo, a
pergunta fica limitada a regiao da Europa. Além disso, para que a visualização nao acontecesse de maneira
muito confusa, para que ela ficasse clara, foram escolhidos os 5 fabricantes mais utilizados e os 3 modelos
mais utilizados por cada fabricante. Se nao houvesse essa limitacao, o grafico ficaria muito poluido, o que
gera um desconforto ao usuario.

Esse grafico, novamente, corrobora a intuição de que na Europa o nivel dos automóveis é maior do que no Brasil.
Observamos que os fabricantes mais comuns la são os mesmos que os presentes no Brasil. Exceto pela BMW, que
muito pouco se observa por aqui, vemos que VW, Ford, Opel (GM) e Toyota são os mais presentes por la.

A difenca de padrao de veículos se observa nos modelos mais comuns, uma vez que os fabricantes são os mesmos.
Por exemplo, os modelos mais comuns da VW são New Beetle, Golf e Polo. Notavelmente, são veículos mais
sofisticados que os modelos comuns brasileiros. O mesmo se observa nos veículos da Ford, cujos modelos
mais populares são Transit, Fusion e Focus. Aqui, esses automóveis são automóveis de elite.

\oneimage{ Observamos que as marcas mais comuns na Europa são as mesmas do brasil, exceto pelo BMW. Nunca vemos BMWs pelo Brasil.}{.99}{sunburst_marca_modelo.png}




%------------------------------------------------------------------------- 
\subsection{Radar}

Para que se possa identificar quais são as marcas e modelos que mais poluem e que menos poluem
de uma forma bastante rica, optou-se por fazer uma visualização em forma de radar. Dessa forma,
foi possivel ver, para cada fabricante, uma serie de atributos que parecem ser correlacionados.
são eles: rotacao do motor (rpm), consumo, velocidade, duracao da viagem, tempo de viagem e 
quantidade de gas carbonico emitido. 

Para que isso pudesse ser sintetizado adequadamente na forma de um radar, foram feitas uma serie
de computacoes em cada um dos itens. Lembrando que diversos desses atributos mencionados, como
co2, rpm, consumo e velocidade mudam periodicamente -- a cada dois segundos --, foi feita uma media
para cada um desses atributos em cada viagem. Posteriormente, foi feita uma normalizacao desses dados,
pois a ferramenta de visualização que o Kibana oferece nao permite que se coloque diferentes escalas
em cada um dos eixos.

Computadas essas medias e normalizados os dados, podemos observar o grafico de radar. E ele nos leva
a algumas informações bastante curiosas:

Peugeot é o fabricante que normalmente se anda com a rpm mais alta e é o cara que tem maior consumo, 
mas nao é o fabricante que anda mais rapido. 

O fabricante que anda mais rapido é o Jeep, e ele consome menos que os Peugeot, normalmente.

Curioso observar tambem, que o Porshe é a marca que se anda com rpm mais baixa -- ao contrario do que normalmente
se imagina -- e cuja velocidade é mais baixa tambem. Acredito que isso se de porque o Porshe deve ter uma
representatividade muito baixa na amostra. Portanto, se voce quiser andar rapido, nao compre um Porshe!

Para quem deseja comprar um automóvel que consuma pouco e que ande com a rotacao do motor baixa, recomenda-se,
de acordo com esse dataset, o uso de um Chevrolet.

\oneimage{ Diversos atributos seccionado por fabricante.}{.99}{radar.png}


O mesmo tipo de visualização pode ser aplicado, nao aos fabricantes, mas aos modelos dos automóveis em si.
Na figura X, observam-se os modelos que menos poluem -- observe que dos 5 apresentados, 2 são chevrolet, o
que esta de acordo com o grafico de radar apresentado por fabricantes -- e observe que os que mais poluem
são modelos esportivos, cuja rotacao de motor normalmente é alta.


\oneimage{ Modelos que menos poluem.}{.99}{radar_modelos_ascending.png}

\oneimage{ Modelos que mais poluem.}{.99}{radar_modelo_descendin.png}









%------------------------------------------------------------------------- 
\subsection{Matriz de correlacao}

Como foi observado no grafico de radar que a rotacao do motor esta intimamente ligada
ao consumo dos automóveis, decidi por fazer uma apresentacao de uma matriz de correlacao. 
O intuito, com essa visualização é verificar quais atributos estao mais ligados um ao
outros. Por exemplo, para ter certeza de que, se o automóvel esta com rotacao maior, que
ele esta poluindo mais. Além disso, para poder observar se faz sentido associar a 
velocidade ah quantidade de emissão de poluentes.

Nessa visualização, ao contrario das anteriores, foi utilizada linguagem R. Ela possui
mais suporte a visualizações e trabalhos estatisticos do que o Kibana. 

Nessa representacao, quanto mais azul e quanto mais cheio esta o circulo superior, maior
é a correlacao entre os atributos. Observamos que quanto maior o consumo de combustivel,
maior é a quantidade de gas carbonico emitido. Observamos tambem que existe grande
correlacao entre rpm e velocidade. 

Um ponto a ser considerado é que parece nao haver correlacao entre velocidade e duracao
de uma viagem. Inicialmente pode-se supor que em viagens mais longas se tende a andar
mais rapidamente. Aqui, entrentanto, esse fato nao pode ser observado.

\oneimage{ Diversos atributos seccionado por fabricante.}{.99}{correlacao.png}


%------------------------------------------------------------------------- 
\subsection{Heatmap com Google Maps}

Para uma análise individual de determinados motoristas, será utilizada novamente a técnica de heatmaps.
Infereriu-se que registros de uma regiao isolada -- como Brasil ou India --, em que existem poucos
registros na base, correspondem ao mesmo motorista.

Com essa hipotese, são plotadas as viagens de um determinado motorista. As regioes mais escuras no
mapa indicam que ele passou por aquele lugar muitas vezes. Regiose claras indicam lugares que ele
frequentou pouco. 

Espera-se dessa maneira, verificar se os caminhos de um determinado motorista são recorrentes ou nao.
Além disso, espera-se verificar se essa técnica é adequada para vislumbrar essa hipotese.

Na primeira figura, observamos que o motorista passou muitas vezes pela mesma rota, dada a linha escura.
Na segunda figura, observamos que o motorista nao repete muitas trajetorias. Na terceira figura, observa-se
que em algumas rotas o motorista usa muitas vezes e em algumas rotas ele frequenta pouco.

Esse terceiro mapa apoia a ideia de que a técnica do heatmap pode ser utilizada com muita eficiencia par
verificar rotas recorrentes, pois claramente se observa pontos escuros que, de fato, correspondem as 
rotas mais utilizadas.

\oneimage{ Linhas escuras indicam rotas que o usuário utilizou muitas vezes.}{.99}{smooth_americano_se_repete_muito.png}

\oneimage{ Linhas claras indicam que o usuário não utilizou muitas vezes a mesma rota.}{.99}{smooth_australiano_nao_se_repete.png}

\oneimage{ Linhas claras e escuras evidenciam quais rotas são as mais utilizadas.}{.99}{smooth_brasileiro.png}



%==========================================
\section{Resultados}
%

As visualizações deram uma visão geral sobre a base de dados, e praticamente todas
as perguntas que puderam ser respondidas. 

\begin{itemize}
  
  \item observamos que, de fato, a maioria dos usuarios esta na Europa. Além disso, observamos
    que por um curto periodo de tempo, houve muita gente utilizando a plataforma, mas que essa
    quantidade de gente a utilizando nao se sustentou por muito tempo.

  \item pode-se observar alguns padrões de comportamento utilizando esse dataset: na Europa, os
    fabricantes mais comuns são praticamente mesmos fabricantes do Brasil. Entretanto, os carros
    comercializados por esses fabricantes por la são superiores aos brasileiros.

  \item observamos de maneira clara, através da visualização de radar, quais são as marcas
  e fabricantes cujos veículos poluem mais e quais poluem menos. Vimos, nesse mesmo grafico de radar,  
  quais são os que gastam mais e os que gastam menos combustivel.

  \item observamos ainda que esse dataset, embora nao seja tao rico quanto se supunha no inicio. Para
  treinamento de modelos de aprendizagem de máquinas, supoe-se que essa base nao seja suficiente

  \item A técnica de heatmap se mostra bastante eficiente par visualizar rotas repetidas, pois deu
  claramente essa nocao ao usuario

  \item Heatmaps realmente ajudam a se ter uma ideia sobre as regioes que possuem maior uso da ferramenta.
  Entretanto, para se ter uma nocao mais exata de proporcoes, foi necessaria uma visualização em pizza.

  \item Os graficos de radar ajudarám a ter claramente uma ideia de muitos atributos sintetizados numa
  pequena regiao espacial.

\end{itemize}




%==========================================
\section{Conclusão}
%

Foi atingido objetivo de se fazer uma exploração básica na base de dados oferecida pelo envirocar. Obteve-se 
intuição sobre essa base de dados e padrões de comportamento dos motoristas de automóvel. 

padrões culturais,
como o tipo de veículos e fabricantes utilizados na Europa em comparacao com o Brasil foram identificados. 
Além disso, padrões individuais de pilotagem de automóveis foram identificados. 

Além disso, foi observadas as distribuicoes temporais e espaciais dos registros de veículos.

A base de dados nao é tao utilizada quanto se supunha -- o autor imaginava que mais gente utilizasse a
plataforma --, mas possui-se uma amostra significativa de pessoas para que se possa levar o trabalho
mais adiante e utilizar essa base para treinar algum modelo de machine learning em trabalhos futuros.

Além disso, pode-se fazer uma análise básica sobre o uso de ferramentas como Kibana e como R para
visualização de informações. O Kibana, mais simples, apresenta graficos mais bonitos. O R, entretanto,
apresenta graficos mais cruz e uma linguagem de baixo nivel. O R, portanto, vai ser mais aproveitado
por usuarios mais experientes -- como programadores --, enquanto o Kibana pode ser utilizado por usuarios
mais leigos.



%==========================================
\iffinal
% use section* for acknowledgement
\section*{References}
%

\begin{enumerate}
  \item JIRKA, S, RAMKE, A, BRORING, A: enviroCar – Crowd Sourced Traffic and Environment Data for Sustainable Mobility
  \item BRORING, A; STASCH, C: enviroCar:  A Citizen Science Platform  for Analyzing and Mapping Crowd-Sourced Car Sensor  Data
  \item LIN, MAIO; JING WEN: Mining GPS data  for mobility  patterns: A survey
  \item VIDAL, WILLIAN. Estudo de viabilidade para detecção de intrusão usando técnicas de Big Data para a análise de logs
  \item VAARANDI, R.; NIZINSKI, P. Comparative analysis of open-source log management solutions for security monitoring and network forensics
\end{enumerate}

\fi



%==========================================

% trigger a \newpage just before the given reference
% number - used to balance the columns on the last page
% adjust value as needed - may need to be readjusted if
% the document is modified later
%\IEEEtriggeratref{8}
% The "triggered" command can be changed if desired:
%\IEEEtriggercmd{\enlargethispage{-5in}}

\bibliographystyle{IEEEtran}
\bibliography{example}

\end{document}
