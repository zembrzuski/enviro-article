\documentclass[10pt, conference]{IEEEtran}

% *** GRAPHICS RELATED PACKAGES ***
\usepackage{subimages}
\setfigdir{figs}

% *** MATH PACKAGES ***
\usepackage[cmex10]{amsmath}
\interdisplaylinepenalty=2500
\usepackage{amsthm}
\newtheorem{definition}{Definition}

% *** PDF, URL AND HYPERLINK PACKAGES ***
\usepackage{hyperref}


\hyphenation{op-tical net-works semi-conduc-tor}


\begin{document}

%
% paper title
% can use linebreaks \\ within to get better formatting as desired
\title{Envirocar visualization \\ exploring an environmental and traffic data set}

%------------------------------------------------------------------------- 
% change the % on next lines to produce the final camera-ready version 
\newif\iffinal
%\finalfalse
\finaltrue
\newcommand{\jemsid}{99999}
%------------------------------------------------------------------------- 

% author names and affiliations
% use a multiple column layout for up to two different
% affiliations

\iffinal
  \author{%
    \IEEEauthorblockN{Rodrigo Claro Zembrzuski}
    \IEEEauthorblockA{%
      Programa de Pos Graduacao em Ciencia da Computacao\\
      Universidade Federal do Rio Grande do Sul - UFRGS\\
      Porto Alegre, Brasil\\
      Web page: \href{http://www.inf.ufrgs.br/~rczembrzuski}{inf.ufrgs.br/$\sim$rczembrzuski}}
  }
\else
  \author{Sibgrapi paper ID: \jemsid \\ }
\fi

% for over three affiliations, or if they all won't fit within the width
% of the page, use this alternative format:
% 
%\author{\IEEEauthorblockN{Michael Shell\IEEEauthorrefmark{1},
%Homer Simpson\IEEEauthorrefmark{2},
%James Kirk\IEEEauthorrefmark{3}, 
%Montgomery Scott\IEEEauthorrefmark{3} and
%Eldon Tyrell\IEEEauthorrefmark{4}}
%\IEEEauthorblockA{\IEEEauthorrefmark{1}School of Electrical and Computer Engineering\\
%Georgia Institute of Technology,
%Atlanta, Georgia 30332--0250\\ Email: see http://www.michaelshell.org/contact.html}
%\IEEEauthorblockA{\IEEEauthorrefmark{2}Twentieth Century Fox, Springfield, USA\\
%Email: homer@thesimpsons.com}
%\IEEEauthorblockA{\IEEEauthorrefmark{3}Starfleet Academy, San Francisco, California 96678-2391\\
%Telephone: (800) 555--1212, Fax: (888) 555--1212}
%\IEEEauthorblockA{\IEEEauthorrefmark{4}Tyrell Inc., 123 Replicant Street, Los Angeles, California 90210--4321}}



%------------------------------------------------------------------------- 
% Special Sibgrapi teaser
%\teaser{%
%  \oneimage{Teasing result of our method: from this data input (left), the relevant feature are extracted using our technique (middle), producing effective result (right).}{.99}{sibgrapi.png}
%}
%------------------------------------------------------------------------- 



% make the title area
\maketitle


\begin{abstract}
Envirocar eh um projeto que coleta e armazena dados ambientais e dados de trafego. 
Sao dados de sensores implantados dentro dos automoveis que trazem informacoes
precisas sobre metricas dos automoveis e dados relevantes ao meio-ambiente, como
emissao de gas-carbonico e consumo do automovel. 

Esse artigo tem por objetivo apresentar uma serie de 
visualizacoes para analise exploratoria desse banco de dados. Com isso, teremos
uma intuicao sobre se esse banco de dados pode servir para alguma coisa, ou nao.

% DO NOT USE SPECIAL CHARACTERS, SYMBOLS, OR MATH IN YOUR TITLE OR ABSTRACT.
%
\end{abstract}

\begin{IEEEkeywords}
envirocar; smartcars; kibana; geolocation; data visualization;

\end{IEEEkeywords}


\IEEEpeerreviewmaketitle


% Wherever Times is specified, Times Roman or Times New Roman may be used. If neither is available on your system, please use the font closest in appearance to Times. Avoid using bit-mapped fonts if possible. True-Type 1 or Open Type fonts are preferred. Please embed symbol fonts, as well, for math, etc.

%==========================================
%==========================================


%==========================================
\section{Introduction}
Envirocar eh um projeto da Universidade de Munster. O objetivo 
do projeto eh oferecer uma plataforma simples para coleta, 
armazenamento e distribuicao de dados ambientais e dados de logistica
de trafego [ref]. 

A plataforma une dados gerados por sensores comuns,
que vem na maioria dos automoveis atuais com sensores de 
geolocalizacao, presente na maioria dos smartphones. Com isso,
cria-se uma base de dados relevante para se medir o fluxo de automoveis
nas avenidas e quantidade de emissao de poluentes nas cidades. 

%------------------------------------------------------------------------- 
\subsection{Motivacao}
%
Um dos grandes desafios da sociedade nos nossos dias eh preservar
e desenvolver a mobilidade urbana de maneira sustentavel, isto eh,
mitigar o impacto do transporte de passageiros meio-ambiente [ref].

Alem disso, observar padroes de comportamento nos motoristas e nas
cidades ao longo do mundo pode ser interessante tanto para ajuda-los
em alguma coisa, quanto para ver diferencas culturais de paises e 
culturas diferrentes evidenciadas no modo em que dirigem.


%\subimages[htb]{Technique overview}{overview}{%
%  \subimage[First step]{.31}{sibgrapi}%
%  \subimage[Second step]{.31}{sibgrapi}%
%  \subimage[Result]{.31}{sibgrapi}%
%}
%------------------------------------------------------------------------- 
\subsection{Objetivos}
%
O objetivo desse trabalho eh fazer uma exploracao basica pela base dados
para ganhar intuicao sobre ela. Para observar se padroes basicos de 
comportamento dos usuarios podem ser detectados conforme ideias
basicas que se tem sobre isso e, por fim, para saber se a base de dados
pode ser utilizada como base para experimentos mais complexos, como 
treinamento de modelos de aprendizagem para um ganho no mundo relevante,
como melhorias no trafego, detecação de anomalias em comportamentos 
e outras cositas mais. 

%------------------------------------------------------------------------- 
\paragraph*{Contributions}
%
Acredito que esse trabalho possa dar um insight interessante sobre
o quao relevante esse dataset pode ser para a observacao de comportamento
dos motoristas.



%------------------------------------------------------------------------- 
\subsection{Resultados esperados}
%
In order to produce this application, we start with this processing, followed by this technique. In order to cope with this challenge, we introduce this formulation to produce this intermediate result. The formulation leads to this type of system, which is efficiently solved by adapting this technique. The final result is produced by this transform. The whole process is schematized in \figref{overview}.


%==========================================
\section{Caracterizacao dos dados}
%
Aqui sera listada, primeiramente, uma descricao sobre a estrutura dos dados apresentados no data set e
sem seguida sera apresentado o problema e questoes que as visualizacoes pretendem responder.

\subsection{Caracterização dos dados}
Cada item do dataset eh composto por dois atributos: 'properties' e 'features'. 

\begin{itemize}

  \item properties: possui as caracteristicas gerais do veiculo em questao. Esta subdividido
  nos seguintes items:
  \begin{itemize}
    \item type -- atributo  categorico -- car or truck
    \item constructionYear -- discreto 
    \item model -- categorico
    \item fuelType -- categorico
    \item engineDisplacement -- discreto
    \item manufacturer -- categorico
  \end{itemize}    

  \item features: possui as caracteristicas do deslocamento -- ou da 'viagem' em si. A viagem
  eh caracterizada pela composicao, a cada dois segundos, do seguinte conjunto de dados:
  \begin{itemize}
    \item coordinates -- geoespacial -- lat/long
    \item speed -- continuo -- km/h
    \item rpm -- continuo -- u/min
    \item gps accuracy -- nao sei -- nao sei
    \item maf -- nao sei -- nao sei
    \item engine load -- nao sei -- nao sei
    \item gps pdop -- nao sei -- nao sei
    \item o2 lambda voltage -- nao sei -- nao sei
    \item throttle position -- nao sei -- nao sei
    \item consumption -- continuo -- l/h
    \item gps vdop -- nao sei -- nao sei
    \item gps speed -- continuo -- km/h
    \item gps bearing -- nao sei -- nao sei
    \item intake pressure -- nao sei -- nao sei
    \item co2 -- atributo continuo -- kg/h
    \item time -- atributo temporal -- timestamp

  \end{itemize}    

\end{itemize}




\subsection{Questoes a serem respondidas}
Existe uma serie de questoes que me vem rah cabeca nesse instante que,
inclusive, ja tomei notas.

\begin{itemize}
  \item quais marcas/modelos consomem mais gasolina?
  \item quais marcas/modelos poluem mais o meio ambiente?
  \item qual o perfil de pilotagem de cada automovel?
  \item muita gente usa esse dataset?
  \item de que lugares sao as pessoas que utilizam esse sistema? no Brasil, no mundo?
  \item existe muita gente utilizando esse sistema hoje em dia?
  \item quais os modelos e fabricantes mais comuns?
  \item os modelos e fabricantes mais comum na Europa sao os mesmos do Brasil?
  \item podemos descobrir quais sao as regioes mais poluidas e regioes menos poluidas?
\end{itemize}




%==========================================
\section{Preparacao dos dados para visualicao}
Aqui eu vou descrever tudo o que fiz para baixar os dados.
Aqui vou explicar por que usar o ElasticSearch tb.



%==========================================
\section{Tecnica de visualização desenvolvida}
\label{sec:technique}
%
A exploracao dos dados foi feito com a ferramenta Kibana. 

Ela, tipicamente, 
eh utilizada para analise por inspeção manual e visualização de informações presentes no
ElasticSearch (MARINO,  2015;  VAARANDI;  PIHELGAS,  2014).   Dessa  forma,  eh  ela  que
ira apresentar os dados armazenados pelo Logstash no ElasticSearch, em uma interface,  via
browser, altamente customizável com histogramas e outros painéis que propiciam uma visão
geral sobre os dados.  O Kibana possibilita transformar os logs em informações úteis (valor)
através  de Dashboards,  pois  permite  realizar  correlação  de  eventos,  filtrar logs
por  origem, hospedeiros, entre outras combinações (VAARANDI; NIZINSKI, 2013)


%------------------------------------------------------------------------- 
\subsection{Histograma}
Com o intuito de saber se muita gente ou se pouca gente utiliza a plataforma envirocar,
foi plotado um histograma. Com ele, alem de saber se muita gente ou se pouca gente o utiliza,
podemos ver a evolucao de sua utilizacao ao longo do tempo.

\oneimage{ No eixo x, observamos a quantidade de uso da plataforma ao longo do tempo.}{.99}{simple_histogram.png}


%------------------------------------------------------------------------- 
\subsection{Heatmaps}

Outra duvida que eu tinha no inicio era a distribuicao geografica das pessoas que utilizam a plataforma. 
Como a plataforma foi desenvolvida na Alemanha, supus que a maioria das pessoas estavam utilizando por lá. 
Estava realmente curioso para saber se existem pessoas em outros lugares do mundo a utilizando tambem. 
Para isso, um heatmap veio a calhar:

\oneimage{ Aqui, podemos observar que a maioria dos usuarios, de fato, se encontra,
na Europa. Contudo, existem pessoas utilizando em diversos outros pontos do globo também.}{.99}{heatmap.png}


%------------------------------------------------------------------------- 
\subsection{Grafico em pizza}

Baseado na visualizacao do heatmap, montei o globo dividido em 6 regioes: America do Norte, America do Sul,
Australia, Europa, India. Com isso, criei um grafico de pizza, que mostra que, de fato, a maioria esmagadora
das pessoas esta na Europa. Existe alguma representatividade tambem na America do Norte e na Australia.

\oneimage{ 98\% dos deslocamentos esta na Europa. 1\% esta na America do Norte e 1\% esta
na Australia. Outras regioes sao sao insignificantes, que nem visualizamos no grafico .}{.99}{pizza_por_regiao.png}



%------------------------------------------------------------------------- 
\subsection{Graficos em pizza}

Uma duvida que eu tive foi quais sao os fabricantes de automovel mais presentes no banco de dados em cada regiao 
do globo. Achei que a divisao por sunburst ficava um pouco complicada de visualizar; entao, fiz n graficos
em pizza, em que cada grafico corresponde a uma regiao do globo e eh exibida a fatia de marcas por regiao.

\oneimage{ Botar uma frase bacana aqui.}{.99}{marcas_por_regiao.png}


%------------------------------------------------------------------------- 
\subsection{Sunburst}

Para responder a pergunta de quais sao os modelos mais utilizados por fabricante, achei bem interessante
de fazer um sunburst. No primeiro nivel, observamos quais sao as 5 marcas mais utilizadas e no segundo nivel
se observam os 3 modelos mais utilizados por cada fabricante. Achei prudente colocar somente 3 modelos no segundo
nivel, pois eu tentei colocar mais e ficou uma coisa muito confusa.

\oneimage{ Botar uma frase bacana aqui.}{.99}{sunburst_marca_modelo.png}



%==========================================
\section{Resultados}
%
Achei interessante que na Europa, a Toyota aparece como uma marca comum. BMW segue sendo top por la.
Fazer mais resultados. Fazer mais resultados. Fazer mais resultados. Fazer mais resultados. Fazer mais resultados. 
Fazer mais resultados. Fazer mais resultados. Fazer mais resultados. Fazer mais resultados. Fazer mais resultados. 
Fazer mais resultados. Fazer mais resultados. Fazer mais resultados. Fazer mais resultados. Fazer mais resultados. 




%==========================================
%\section{Experiments}
%
%We validate our technique through a series of experiments.


%------------------------------------------------------------------------- 
%\paragraph*{First experiment}
%
%The first experiment checks this aspect of our method on perfect examples.


%------------------------------------------------------------------------- 
%\paragraph*{Second experiment}
%
%The second experiment checks the speedup obtained by the implementation strategy compared to previous technique~\cite{Sibgrapi2015}.


%------------------------------------------------------------------------- 
%\paragraph*{Third experiment}
%
%The last experiment test our method on real data.



%==========================================
%\section{Results and Discussion}
%
%We performed the above-mentioned experiments on the following type of data: \ldots{} For each data, we used the following tuning parameters of our method.


%\begin{table}
%\caption{Performances results: timings are expressed in milliseconds.}
%\label{tab:perfs}
%\centering
%\begin{tabular}{lr|rr|c}
%\multicolumn{1}{c}{\bf Data} &
%\multicolumn{1}{c|}{\bf Size} &
%\multicolumn{1}{c}{\bf Ours} &
%\multicolumn{1}{c|}{\bf Previous} &
%\multicolumn{1}{c}{\bf Gain} \\ \hline
%Data 1	&        50 	& 0.1 &     1 000	& x$10^3$ \\
%Data 2	&      100 	& 0.2 &     2 000	& x$10^3$ \\
%Data 3	&      500 	& 0.8 &   10 000	& x$10^3$ \\
%Data 4	&   1 000 	& 1.2 &   20 000	& x$10^3$ \\
%Data 5	&   5 000 	& 1.9 & 100 000	& x$10^4$ \\
%Data 6	& 10 000 	& 2.1 & 200 000	& x$10^4$
%\end{tabular}
%\end{table}
%
%------------------------------------------------------------------------- 
%\subsection{Performances}
%%
%We report on Table~\ref{tab:perfs} the performances of our technique on a computer at xxGhz with this graphic card.
%We observe that our technique outperforms previous approaches on this kind of data, and an equivalent result on this other kind of data.

%\subimages[htb]{Quality assessment}{quality}{
%  \subimage{.48}{sibgrapi}%
%  \subimage{.48}{sibgrapi}%
%}


%------------------------------------------------------------------------- 
%\subsection{Quality}
%
%As observed on \figref{quality}, our method achieve good results in this situation. 
%This can be measured by this criterion, and the results are reported on Table~\ref{tab:quality}.

%\begin{table}
%\caption{Quality measures: timings are expressed in milliseconds.}
%\label{tab:quality}
%\centering
%\begin{tabular}{l|r|r}
%\multicolumn{1}{c}{\bf Images} &
%\multicolumn{1}{c|}{\bf PSNR} &
%\multicolumn{1}{c}{\bf  MSE} \\ \hline
%Image 1	&  40.2	& 0.02 \\
%Image 2	&  30.9	& 1.02 \\
%Image 3 &  20.1 & 0.18 \\
%\end{tabular}
%\end{table}


%------------------------------------------------------------------------- 
%\subsection{Limitation}
%
%As mentioned in Section~\ref{sec:technique}, we expect our method to suit better this kind of data. On the other kind, this particularity does not fit into our formulation for this and that reason. Indeed, this can be observed in the results of \figref{quality}. 
%We plan to improve for that kind of data in future work. However, our technique performed well on this data, which does not respect our condition, since this other aspect reduced the negative impact of its characteristic.


%==========================================
\section{Conclusao}
%
Aqui vai minha conclusao. Aqui vai minha conclusao. Aqui vai minha conclusao. Aqui vai minha conclusao. Aqui vai minha conclusao. Aqui vai minha conclusao. Aqui vai minha conclusao. Aqui vai minha conclusao. Aqui vai minha conclusao. Aqui vai minha conclusao. Aqui vai minha conclusao. Aqui vai minha conclusao. Aqui vai minha conclusao. Aqui vai minha conclusao. Aqui vai minha conclusao. Aqui vai minha conclusao. Aqui vai minha conclusao. Aqui vai minha conclusao. Aqui vai minha conclusao. Aqui vai minha conclusao. Aqui vai minha conclusao. Aqui vai minha conclusao. Aqui vai minha conclusao. Aqui vai minha conclusao. Aqui vai minha conclusao. Aqui vai minha conclusao. Aqui vai minha conclusao. Aqui vai minha conclusao. Aqui vai minha conclusao. Aqui vai minha conclusao.


%==========================================
\iffinal
% use section* for acknowledgement
\section*{Acknowledgment}
%
The authors would like to thank this colleague and this financing institute.
\fi



%==========================================

% trigger a \newpage just before the given reference
% number - used to balance the columns on the last page
% adjust value as needed - may need to be readjusted if
% the document is modified later
%\IEEEtriggeratref{8}
% The "triggered" command can be changed if desired:
%\IEEEtriggercmd{\enlargethispage{-5in}}

\bibliographystyle{IEEEtran}
\bibliography{example}

\end{document}
